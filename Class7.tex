% Options for packages loaded elsewhere
\PassOptionsToPackage{unicode}{hyperref}
\PassOptionsToPackage{hyphens}{url}
%
\documentclass[
]{article}
\usepackage{lmodern}
\usepackage{amssymb,amsmath}
\usepackage{ifxetex,ifluatex}
\ifnum 0\ifxetex 1\fi\ifluatex 1\fi=0 % if pdftex
  \usepackage[T1]{fontenc}
  \usepackage[utf8]{inputenc}
  \usepackage{textcomp} % provide euro and other symbols
\else % if luatex or xetex
  \usepackage{unicode-math}
  \defaultfontfeatures{Scale=MatchLowercase}
  \defaultfontfeatures[\rmfamily]{Ligatures=TeX,Scale=1}
\fi
% Use upquote if available, for straight quotes in verbatim environments
\IfFileExists{upquote.sty}{\usepackage{upquote}}{}
\IfFileExists{microtype.sty}{% use microtype if available
  \usepackage[]{microtype}
  \UseMicrotypeSet[protrusion]{basicmath} % disable protrusion for tt fonts
}{}
\makeatletter
\@ifundefined{KOMAClassName}{% if non-KOMA class
  \IfFileExists{parskip.sty}{%
    \usepackage{parskip}
  }{% else
    \setlength{\parindent}{0pt}
    \setlength{\parskip}{6pt plus 2pt minus 1pt}}
}{% if KOMA class
  \KOMAoptions{parskip=half}}
\makeatother
\usepackage{xcolor}
\IfFileExists{xurl.sty}{\usepackage{xurl}}{} % add URL line breaks if available
\IfFileExists{bookmark.sty}{\usepackage{bookmark}}{\usepackage{hyperref}}
\hypersetup{
  pdftitle={Class 7},
  hidelinks,
  pdfcreator={LaTeX via pandoc}}
\urlstyle{same} % disable monospaced font for URLs
\usepackage[margin=1in]{geometry}
\usepackage{color}
\usepackage{fancyvrb}
\newcommand{\VerbBar}{|}
\newcommand{\VERB}{\Verb[commandchars=\\\{\}]}
\DefineVerbatimEnvironment{Highlighting}{Verbatim}{commandchars=\\\{\}}
% Add ',fontsize=\small' for more characters per line
\usepackage{framed}
\definecolor{shadecolor}{RGB}{248,248,248}
\newenvironment{Shaded}{\begin{snugshade}}{\end{snugshade}}
\newcommand{\AlertTok}[1]{\textcolor[rgb]{0.94,0.16,0.16}{#1}}
\newcommand{\AnnotationTok}[1]{\textcolor[rgb]{0.56,0.35,0.01}{\textbf{\textit{#1}}}}
\newcommand{\AttributeTok}[1]{\textcolor[rgb]{0.77,0.63,0.00}{#1}}
\newcommand{\BaseNTok}[1]{\textcolor[rgb]{0.00,0.00,0.81}{#1}}
\newcommand{\BuiltInTok}[1]{#1}
\newcommand{\CharTok}[1]{\textcolor[rgb]{0.31,0.60,0.02}{#1}}
\newcommand{\CommentTok}[1]{\textcolor[rgb]{0.56,0.35,0.01}{\textit{#1}}}
\newcommand{\CommentVarTok}[1]{\textcolor[rgb]{0.56,0.35,0.01}{\textbf{\textit{#1}}}}
\newcommand{\ConstantTok}[1]{\textcolor[rgb]{0.00,0.00,0.00}{#1}}
\newcommand{\ControlFlowTok}[1]{\textcolor[rgb]{0.13,0.29,0.53}{\textbf{#1}}}
\newcommand{\DataTypeTok}[1]{\textcolor[rgb]{0.13,0.29,0.53}{#1}}
\newcommand{\DecValTok}[1]{\textcolor[rgb]{0.00,0.00,0.81}{#1}}
\newcommand{\DocumentationTok}[1]{\textcolor[rgb]{0.56,0.35,0.01}{\textbf{\textit{#1}}}}
\newcommand{\ErrorTok}[1]{\textcolor[rgb]{0.64,0.00,0.00}{\textbf{#1}}}
\newcommand{\ExtensionTok}[1]{#1}
\newcommand{\FloatTok}[1]{\textcolor[rgb]{0.00,0.00,0.81}{#1}}
\newcommand{\FunctionTok}[1]{\textcolor[rgb]{0.00,0.00,0.00}{#1}}
\newcommand{\ImportTok}[1]{#1}
\newcommand{\InformationTok}[1]{\textcolor[rgb]{0.56,0.35,0.01}{\textbf{\textit{#1}}}}
\newcommand{\KeywordTok}[1]{\textcolor[rgb]{0.13,0.29,0.53}{\textbf{#1}}}
\newcommand{\NormalTok}[1]{#1}
\newcommand{\OperatorTok}[1]{\textcolor[rgb]{0.81,0.36,0.00}{\textbf{#1}}}
\newcommand{\OtherTok}[1]{\textcolor[rgb]{0.56,0.35,0.01}{#1}}
\newcommand{\PreprocessorTok}[1]{\textcolor[rgb]{0.56,0.35,0.01}{\textit{#1}}}
\newcommand{\RegionMarkerTok}[1]{#1}
\newcommand{\SpecialCharTok}[1]{\textcolor[rgb]{0.00,0.00,0.00}{#1}}
\newcommand{\SpecialStringTok}[1]{\textcolor[rgb]{0.31,0.60,0.02}{#1}}
\newcommand{\StringTok}[1]{\textcolor[rgb]{0.31,0.60,0.02}{#1}}
\newcommand{\VariableTok}[1]{\textcolor[rgb]{0.00,0.00,0.00}{#1}}
\newcommand{\VerbatimStringTok}[1]{\textcolor[rgb]{0.31,0.60,0.02}{#1}}
\newcommand{\WarningTok}[1]{\textcolor[rgb]{0.56,0.35,0.01}{\textbf{\textit{#1}}}}
\usepackage{graphicx,grffile}
\makeatletter
\def\maxwidth{\ifdim\Gin@nat@width>\linewidth\linewidth\else\Gin@nat@width\fi}
\def\maxheight{\ifdim\Gin@nat@height>\textheight\textheight\else\Gin@nat@height\fi}
\makeatother
% Scale images if necessary, so that they will not overflow the page
% margins by default, and it is still possible to overwrite the defaults
% using explicit options in \includegraphics[width, height, ...]{}
\setkeys{Gin}{width=\maxwidth,height=\maxheight,keepaspectratio}
% Set default figure placement to htbp
\makeatletter
\def\fps@figure{htbp}
\makeatother
\setlength{\emergencystretch}{3em} % prevent overfull lines
\providecommand{\tightlist}{%
  \setlength{\itemsep}{0pt}\setlength{\parskip}{0pt}}
\setcounter{secnumdepth}{-\maxdimen} % remove section numbering

\title{Class 7}
\author{}
\date{\vspace{-2.5em}}

\begin{document}
\maketitle

\hypertarget{required-files}{%
\subsection{Required files}\label{required-files}}

\begin{enumerate}
\def\labelenumi{\arabic{enumi}.}
\tightlist
\item
  ado-actstabled.RDS
\end{enumerate}

\hypertarget{import}{%
\subsection{Import}\label{import}}

Lots of people are having trouble getting the RDS objects imported, but
it can by done by making the RDS from the script

\begin{Shaded}
\begin{Highlighting}[]
\NormalTok{knitr}\OperatorTok{::}\NormalTok{opts_chunk}\OperatorTok{$}\KeywordTok{set}\NormalTok{(}\DataTypeTok{echo =} \OtherTok{TRUE}\NormalTok{)}

\NormalTok{speakerraw.l<-}\KeywordTok{readRDS}\NormalTok{(}\StringTok{"ado-speakerrawlist.RDS"}\NormalTok{)}
\NormalTok{actfreqraw.l <-}\StringTok{ }\KeywordTok{readRDS}\NormalTok{(}\StringTok{"ado-actstabled.RDS"}\NormalTok{)}
\KeywordTok{str}\NormalTok{(actfreqraw.l)}
\end{Highlighting}
\end{Shaded}

\begin{verbatim}
## List of 5
##  $ act1: 'table' int [1:902(1d)] 95 1 2 1 1 1 1 2 1 1 ...
##   ..- attr(*, "dimnames")=List of 1
##   .. ..$ actwords: chr [1:902] "a" "abides" "about" "accordant" ...
##  $ act2: 'table' int [1:1190(1d)] 126 1 3 1 1 1 1 1 2 1 ...
##   ..- attr(*, "dimnames")=List of 1
##   .. ..$ actwords: chr [1:1190] "a" "abhor" "about" "absent" ...
##  $ act3: 'table' int [1:1091(1d)] 94 3 1 1 2 1 1 2 1 1 ...
##   ..- attr(*, "dimnames")=List of 1
##   .. ..$ actwords: chr [1:1091] "a" "about" "accordingly" "acquaint" ...
##  $ act4: 'table' int [1:849(1d)] 61 1 5 1 2 2 4 1 1 1 ...
##   ..- attr(*, "dimnames")=List of 1
##   .. ..$ actwords: chr [1:849] "a" "ability" "about" "account" ...
##  $ act5: 'table' int [1:1156(1d)] 85 3 1 1 1 2 1 1 1 1 ...
##   ..- attr(*, "dimnames")=List of 1
##   .. ..$ actwords: chr [1:1156] "a" "about" "above" "abused" ...
\end{verbatim}

\begin{Shaded}
\begin{Highlighting}[]
\NormalTok{actfreqraw.l[[}\DecValTok{1}\NormalTok{]][}\DecValTok{1}\OperatorTok{:}\DecValTok{10}\NormalTok{]}
\end{Highlighting}
\end{Shaded}

\begin{verbatim}
## actwords
##           a      abides       about   accordant    achiever acknowledge 
##          95           1           2           1           1           1 
##    acquaint      action        adam      affect 
##           1           2           1           1
\end{verbatim}

\hypertarget{matracies-and-data-frames}{%
\subsection{Matracies and Data frames}\label{matracies-and-data-frames}}

Sum adds things together

lapply first applies the funtion to the object returning the output as a
list always returns a list

sapply returns a vector s stands for simplify

there is also just apply

do.call also can take a function and apply to a data structure Tries to
bring all of the data together in some way applies to the list of the
whole

apply vs loops very similar, but not the same R prefers apply apply is
built to work with vectorized data

matracies have 2 demensions rbind makes matrixes RowBind

R does rows then columns

\begin{Shaded}
\begin{Highlighting}[]
\KeywordTok{sum}\NormalTok{(}\DecValTok{1}\NormalTok{,}\DecValTok{3}\NormalTok{) }\CommentTok{#Adds things together}
\end{Highlighting}
\end{Shaded}

\begin{verbatim}
## [1] 4
\end{verbatim}

\begin{Shaded}
\begin{Highlighting}[]
\NormalTok{mylist <-}\StringTok{ }\KeywordTok{list}\NormalTok{(}\KeywordTok{c}\NormalTok{(}\DecValTok{1}\NormalTok{,}\DecValTok{2}\NormalTok{,}\DecValTok{3}\NormalTok{), }\KeywordTok{c}\NormalTok{(}\DecValTok{4}\NormalTok{,}\DecValTok{5}\NormalTok{,}\DecValTok{6}\NormalTok{))}
\KeywordTok{sum}\NormalTok{(mylist[[}\DecValTok{1}\NormalTok{]]) }\CommentTok{# 6}
\end{Highlighting}
\end{Shaded}

\begin{verbatim}
## [1] 6
\end{verbatim}

\begin{Shaded}
\begin{Highlighting}[]
\CommentTok{#lapply}
\KeywordTok{lapply}\NormalTok{(mylist, sum)}
\end{Highlighting}
\end{Shaded}

\begin{verbatim}
## [[1]]
## [1] 6
## 
## [[2]]
## [1] 15
\end{verbatim}

\begin{Shaded}
\begin{Highlighting}[]
\CommentTok{#sapply}
\KeywordTok{sapply}\NormalTok{(mylist, sum)}
\end{Highlighting}
\end{Shaded}

\begin{verbatim}
## [1]  6 15
\end{verbatim}

\begin{Shaded}
\begin{Highlighting}[]
\KeywordTok{do.call}\NormalTok{(sum, mylist)}
\end{Highlighting}
\end{Shaded}

\begin{verbatim}
## [1] 21
\end{verbatim}

\begin{Shaded}
\begin{Highlighting}[]
\KeywordTok{rbind}\NormalTok{(}\DecValTok{1}\OperatorTok{:}\DecValTok{3}\NormalTok{, }\DecValTok{4}\OperatorTok{:}\DecValTok{6}\NormalTok{)}
\end{Highlighting}
\end{Shaded}

\begin{verbatim}
##      [,1] [,2] [,3]
## [1,]    1    2    3
## [2,]    4    5    6
\end{verbatim}

\begin{Shaded}
\begin{Highlighting}[]
\CommentTok{#To make a matrix}
\NormalTok{mymatrix <-}\StringTok{ }\KeywordTok{rbind}\NormalTok{(}\DecValTok{1}\OperatorTok{:}\DecValTok{3}\NormalTok{, }\DecValTok{4}\OperatorTok{:}\DecValTok{6}\NormalTok{)}

\CommentTok{#How to get 5 back}
\NormalTok{mymatrix[}\DecValTok{2}\NormalTok{,}\DecValTok{2}\NormalTok{]}
\end{Highlighting}
\end{Shaded}

\begin{verbatim}
## [1] 5
\end{verbatim}

\begin{Shaded}
\begin{Highlighting}[]
\CommentTok{#Does not work, out of bounds.}
\CommentTok{# mymatrix[1,4]}
\CommentTok{# Error in mymatrix[1, 4] : subscript out of bounds}
\end{Highlighting}
\end{Shaded}

\hypertarget{cbind}{%
\subsection{Cbind}\label{cbind}}

cbind - also makes matricies column bind

\begin{Shaded}
\begin{Highlighting}[]
\NormalTok{mymatrix2 <-}\StringTok{ }\KeywordTok{cbind}\NormalTok{(}\DecValTok{1}\OperatorTok{:}\DecValTok{3}\NormalTok{, }\DecValTok{4}\OperatorTok{:}\DecValTok{6}\NormalTok{)}
\NormalTok{mymatrix2[}\DecValTok{2}\NormalTok{,}\DecValTok{2}\NormalTok{]}
\end{Highlighting}
\end{Shaded}

\begin{verbatim}
## [1] 5
\end{verbatim}

\hypertarget{bigger-matrix-time}{%
\subsection{Bigger matrix time}\label{bigger-matrix-time}}

Missmatched column lengths repeat the shorter column

\begin{Shaded}
\begin{Highlighting}[]
\NormalTok{mymatrix <-}\StringTok{ }\KeywordTok{rbind}\NormalTok{(}\DecValTok{5}\OperatorTok{:}\DecValTok{10}\NormalTok{, }\DecValTok{16}\OperatorTok{:}\DecValTok{21}\NormalTok{, }\DecValTok{100}\OperatorTok{:}\DecValTok{105}\NormalTok{)}
\NormalTok{mymatrix}
\end{Highlighting}
\end{Shaded}

\begin{verbatim}
##      [,1] [,2] [,3] [,4] [,5] [,6]
## [1,]    5    6    7    8    9   10
## [2,]   16   17   18   19   20   21
## [3,]  100  101  102  103  104  105
\end{verbatim}

\begin{Shaded}
\begin{Highlighting}[]
\KeywordTok{apply}\NormalTok{(mymatrix, }\DecValTok{1}\NormalTok{, nchar)}
\end{Highlighting}
\end{Shaded}

\begin{verbatim}
##      [,1] [,2] [,3]
## [1,]    1    2    3
## [2,]    1    2    3
## [3,]    1    2    3
## [4,]    1    2    3
## [5,]    1    2    3
## [6,]    2    2    3
\end{verbatim}

\begin{Shaded}
\begin{Highlighting}[]
\NormalTok{somenums <-}\StringTok{ }\KeywordTok{cbind}\NormalTok{(}\DataTypeTok{a=}\KeywordTok{c}\NormalTok{(}\DecValTok{3}\NormalTok{, }\DecValTok{2}\NormalTok{, }\DecValTok{4}\NormalTok{, }\DecValTok{5}\NormalTok{), }\DataTypeTok{b =} \DecValTok{1}\OperatorTok{:}\DecValTok{15}\NormalTok{) }\CommentTok{# Mismatched column lengths, but it still works}
\end{Highlighting}
\end{Shaded}

\begin{verbatim}
## Warning in cbind(a = c(3, 2, 4, 5), b = 1:15): number of rows of result is not a
## multiple of vector length (arg 1)
\end{verbatim}

\begin{Shaded}
\begin{Highlighting}[]
\NormalTok{somenums <-}\StringTok{ }\KeywordTok{cbind}\NormalTok{(}\DataTypeTok{a=}\KeywordTok{c}\NormalTok{(}\DecValTok{3}\NormalTok{, }\DecValTok{2}\NormalTok{, }\DecValTok{4}\NormalTok{, }\DecValTok{5}\NormalTok{), }\DataTypeTok{b =} \DecValTok{1}\OperatorTok{:}\DecValTok{4}\NormalTok{)}
\NormalTok{somenums}
\end{Highlighting}
\end{Shaded}

\begin{verbatim}
##      a b
## [1,] 3 1
## [2,] 2 2
## [3,] 4 3
## [4,] 5 4
\end{verbatim}

\begin{Shaded}
\begin{Highlighting}[]
\KeywordTok{apply}\NormalTok{(somenums, }\DecValTok{1}\NormalTok{, sum)}
\end{Highlighting}
\end{Shaded}

\begin{verbatim}
## [1] 4 4 7 9
\end{verbatim}

\hypertarget{lapply}{%
\subsection{lapply}\label{lapply}}

lapply() using ``{[}'', the select operator - this is bad

\hypertarget{anewlist}{%
\subsection{anewlist}\label{anewlist}}

matrix is good enough to auto wrap to get an item from the matrix
coordinates can be used or a list style position can be used.
matrix3{[}5{]} goes through the matrix till it hits the 5th item.

lapply(anewlist, ``{[}'',,2) This is just going to take the second
column out This gives a warning, but it runs

\begin{Shaded}
\begin{Highlighting}[]
\NormalTok{anewlist <-}\StringTok{ }\KeywordTok{list}\NormalTok{(}\KeywordTok{matrix}\NormalTok{(}\DecValTok{1}\OperatorTok{:}\DecValTok{9}\NormalTok{, }\DecValTok{3}\NormalTok{,}\DecValTok{3}\NormalTok{),}\KeywordTok{matrix}\NormalTok{(}\DecValTok{4}\OperatorTok{:}\DecValTok{15}\NormalTok{, }\DecValTok{3}\NormalTok{, }\DecValTok{3}\NormalTok{), }\KeywordTok{cbind}\NormalTok{(}\DecValTok{8}\OperatorTok{:}\DecValTok{10}\NormalTok{, }\DecValTok{8}\OperatorTok{:}\DecValTok{10}\NormalTok{))}
\KeywordTok{str}\NormalTok{(anewlist)}
\end{Highlighting}
\end{Shaded}

\begin{verbatim}
## List of 3
##  $ : int [1:3, 1:3] 1 2 3 4 5 6 7 8 9
##  $ : int [1:3, 1:3] 4 5 6 7 8 9 10 11 12
##  $ : int [1:3, 1:2] 8 9 10 8 9 10
\end{verbatim}

\begin{Shaded}
\begin{Highlighting}[]
\NormalTok{mymatrix3 <-}\StringTok{ }\KeywordTok{matrix}\NormalTok{(}\DecValTok{1}\OperatorTok{:}\DecValTok{9}\NormalTok{, }\DecValTok{3}\NormalTok{,}\DecValTok{3}\NormalTok{)}
\NormalTok{mymatrix3}
\end{Highlighting}
\end{Shaded}

\begin{verbatim}
##      [,1] [,2] [,3]
## [1,]    1    4    7
## [2,]    2    5    8
## [3,]    3    6    9
\end{verbatim}

\begin{Shaded}
\begin{Highlighting}[]
\NormalTok{mymatrix3[}\DecValTok{2}\NormalTok{,}\DecValTok{2}\NormalTok{]}
\end{Highlighting}
\end{Shaded}

\begin{verbatim}
## [1] 5
\end{verbatim}

\begin{Shaded}
\begin{Highlighting}[]
\NormalTok{mymatrix3[}\DecValTok{5}\NormalTok{]}
\end{Highlighting}
\end{Shaded}

\begin{verbatim}
## [1] 5
\end{verbatim}

\begin{Shaded}
\begin{Highlighting}[]
\KeywordTok{lapply}\NormalTok{(anewlist, }\StringTok{"["}\NormalTok{,,}\DecValTok{2}\NormalTok{)}
\end{Highlighting}
\end{Shaded}

\begin{verbatim}
## [[1]]
## [1] 4 5 6
## 
## [[2]]
## [1] 7 8 9
## 
## [[3]]
## [1]  8  9 10
\end{verbatim}

\hypertarget{english-major-stuff}{%
\subsection{English major stuff}\label{english-major-stuff}}

It will go act by act and sum the words in each act

lapply will return a list

sapply returns a more compact vector

is.na helps you look for nothings so you can zap it/ replace it with 0

\begin{Shaded}
\begin{Highlighting}[]
\KeywordTok{lapply}\NormalTok{(actfreqraw.l, sum)}
\end{Highlighting}
\end{Shaded}

\begin{verbatim}
## $act1
## [1] 3286
## 
## $act2
## [1] 5536
## 
## $act3
## [1] 4407
## 
## $act4
## [1] 3171
## 
## $act5
## [1] 4697
\end{verbatim}

\begin{Shaded}
\begin{Highlighting}[]
\KeywordTok{sapply}\NormalTok{(actfreqraw.l, sum)}
\end{Highlighting}
\end{Shaded}

\begin{verbatim}
## act1 act2 act3 act4 act5 
## 3286 5536 4407 3171 4697
\end{verbatim}

\begin{Shaded}
\begin{Highlighting}[]
\NormalTok{actfreqraw.l[[}\DecValTok{1}\NormalTok{]][}\StringTok{"nothing"}\NormalTok{] }\CommentTok{# Looks for how many times nothing appears in each act}
\end{Highlighting}
\end{Shaded}

\begin{verbatim}
## nothing 
##       1
\end{verbatim}

\begin{Shaded}
\begin{Highlighting}[]
\NormalTok{nothings.m <-}\StringTok{ }\KeywordTok{sapply}\NormalTok{(actfreqraw.l, }\StringTok{"["}\NormalTok{, }\KeywordTok{c}\NormalTok{(}\StringTok{"never"}\NormalTok{, }\StringTok{"none"}\NormalTok{, }\StringTok{"not"}\NormalTok{, }\StringTok{"nothing"}\NormalTok{))}
\NormalTok{nothings.m}
\end{Highlighting}
\end{Shaded}

\begin{verbatim}
##         act1 act2 act3 act4 act5
## never      5   15    7    2   10
## none       5    5    5    8   NA
## not       33   61   54   55   47
## nothing    1    3    3    5    8
\end{verbatim}

\begin{Shaded}
\begin{Highlighting}[]
\KeywordTok{is.na}\NormalTok{(nothings.m) }\CommentTok{# Looks for NAs}
\end{Highlighting}
\end{Shaded}

\begin{verbatim}
##          act1  act2  act3  act4  act5
## never   FALSE FALSE FALSE FALSE FALSE
## none    FALSE FALSE FALSE FALSE  TRUE
## not     FALSE FALSE FALSE FALSE FALSE
## nothing FALSE FALSE FALSE FALSE FALSE
\end{verbatim}

\begin{Shaded}
\begin{Highlighting}[]
\KeywordTok{which}\NormalTok{(}\KeywordTok{is.na}\NormalTok{(nothings.m))}
\end{Highlighting}
\end{Shaded}

\begin{verbatim}
## [1] 18
\end{verbatim}

\begin{Shaded}
\begin{Highlighting}[]
\NormalTok{nothings.m[}\DecValTok{18}\NormalTok{] <-}\StringTok{ }\DecValTok{0}
\NormalTok{nothings.m}
\end{Highlighting}
\end{Shaded}

\begin{verbatim}
##         act1 act2 act3 act4 act5
## never      5   15    7    2   10
## none       5    5    5    8    0
## not       33   61   54   55   47
## nothing    1    3    3    5    8
\end{verbatim}

\hypertarget{looking-for-character-names}{%
\subsection{Looking for character
names}\label{looking-for-character-names}}

actfreqraw.l{[}{[}1{]}{]}{[}``beatrice''{]} looks for where beatrices
name is used by the characters it is not used in the first act

NA still infects things, so we have to fix things

\begin{Shaded}
\begin{Highlighting}[]
\NormalTok{actfreqraw.l[[}\DecValTok{1}\NormalTok{]][}\StringTok{"beatrice"}\NormalTok{]}
\end{Highlighting}
\end{Shaded}

\begin{verbatim}
## <NA> 
##   NA
\end{verbatim}

\begin{Shaded}
\begin{Highlighting}[]
\KeywordTok{unlist}\NormalTok{(}\KeywordTok{lapply}\NormalTok{(actfreqraw.l, }\StringTok{'['}\NormalTok{, }\StringTok{"beatrice"}\NormalTok{))}
\end{Highlighting}
\end{Shaded}

\begin{verbatim}
##       act1.NA act2.beatrice act3.beatrice act4.beatrice act5.beatrice 
##            NA            11            14             8            10
\end{verbatim}

\begin{Shaded}
\begin{Highlighting}[]
\CommentTok{# The column name gets infected by NA}

\NormalTok{ladies.m <-}\StringTok{ }\KeywordTok{do.call}\NormalTok{(rbind, }\KeywordTok{lapply}\NormalTok{(actfreqraw.l, }\StringTok{"["}\NormalTok{, }\KeywordTok{c}\NormalTok{(}\StringTok{"beatrice"}\NormalTok{, }\StringTok{"hero"}\NormalTok{)))}
\KeywordTok{colnames}\NormalTok{(ladies.m)}
\end{Highlighting}
\end{Shaded}

\begin{verbatim}
## [1] NA     "hero"
\end{verbatim}

\begin{Shaded}
\begin{Highlighting}[]
\KeywordTok{colnames}\NormalTok{(ladies.m)[}\DecValTok{1}\NormalTok{]<-}\StringTok{"beatrice"}
\KeywordTok{colnames}\NormalTok{(ladies.m)}
\end{Highlighting}
\end{Shaded}

\begin{verbatim}
## [1] "beatrice" "hero"
\end{verbatim}

\hypertarget{homework-excersize-related-stuff}{%
\subsection{Homework excersize related
stuff}\label{homework-excersize-related-stuff}}

this thing is called a term document matrix

NOTE: Document term matrixes are different

\begin{Shaded}
\begin{Highlighting}[]
\NormalTok{nothings.m <-}\StringTok{ }\KeywordTok{sapply}\NormalTok{(actfreqraw.l, }\StringTok{'['}\NormalTok{, }\KeywordTok{c}\NormalTok{(}\StringTok{"never"}\NormalTok{, }\StringTok{"none"}\NormalTok{, }\StringTok{"not"}\NormalTok{, }\StringTok{"nothing"}\NormalTok{))}
\NormalTok{nothings.m[}\DecValTok{18}\NormalTok{] <-}\StringTok{ }\DecValTok{0}
\end{Highlighting}
\end{Shaded}

\end{document}
